%\begingroup
%\let\cleardoublepage\clearpage


% French abstract
\cleardoublepage
\chapter*{Résumé}
\markboth{Résumé}{Résumé}
\addcontentsline{toc}{chapter}{Résumé} % adds an entry to the table of contents

	L'hypertension artérielle (HTA) est une des maladies cardiovasculaires les plus répandues. Aujourd'hui, nous connaissons les facteurs d'influence principaux de l'HTA mais le rôle des stéroïdes n'a été que peu exploré. Ce travail, fait en collaboration avec l'Institut universitaire de médecine sociale et préventive (IUMSP) de Lausanne et l'institut Computational Intelligence for Computational Biology (CI4CB) d'Yverdon-les-Bain à comme objectif d'obtenir une meilleur compréhension du rôles des stéroïdes pour l'HTA.

	Pour atteindre cet objectif, nous avons analysé les mesures stéroïdiennes de jour obtenue dans le dataset de la cohorte SKIPOGH. Un prétraitement des donnée a réduit le dataset de 1129 patient à 1067 et les stéroïdes de 40 à 35. La méthode d'analyse a ensuite été divisée en 2 parties. La première est une analyse des corrélations globale à l'aide d'une matrice de corrélation. La seconde est une analyse des corrélations partielles à partir de visualisations d'une self-organizing map (SOM) et une coloration de SOM: l'âge, le sex, le BMI, la pression artérielle diastolique et systolique. 

	%TODO Finir en ajoutant les résultats et quelques conclusions de l'analyse.

	\vskip0.5cm

	\subsubsection*{Mots-clés} 
	
	\vskip0.25cm
	
	\noindent stéroïdes, corrélations, self-organizing map, SOM, SKIPOGH, IUMSP, CI4CB