% =============================================================================
\chapter{Introduction}
% =============================================================================

	% l'intro c'est 1/7 de ton travail :-)
	% Présentation des personnes et organismes impliqués.
	% Présenter la problématique de l'IUMSP
		% Mieux comprendre quels sont les rôles des stéroïdes pour l'HTA
		% Pouvoir prédire les crises hypertensive
	
	L'hypertension artérielle (HTA) est l'augmentation de la pression artérielle. C'est une des maladies cardiovasculaires les plus répandues. Elle favorise la survenue d’accidents cardiaques, vasculaires cérébraux (AVC) ou encore d'insuffisances rénales.

	En 2005, Kearney et ses collègues \cite{Kearney:2005} ont publié une meta-analyse qui fait le bilan de l'état de la maladie au niveau mondial. Ses résultats sont clairs, ce problème de santé publique nécessite une attention prioritaire. En effet, dans les années 2000, la prévalence mondiale de cette maladie cardiovasculaire comptait 972 millions d'adultes. Selon l'Office fédéral de la statistique \cite{OFS:HTA}, en 2012, 27\% de la population suisse est diagnostiquée positivement à l'HTA, 7\% de plus qu'en 1997.

	Aujourd'hui, nous connaissons les facteurs d'influence de l'HTA. Les études mettent en exergue: le sexe, l'âge, l'Indice de Masse Corporelle (BMI), la consommation d'alcool, la consommation de tabac, une nutrition trop salée, une activité physique trop faible ou encore l'hérédité. % TODO : Mettre une référence 

	L'Institut universitaire de médecine sociale et préventive\footnote{\url{https://www.iumsp.ch}} (IUMSP) de Lausanne mène le projet de recherche suisse sur la génétique de l’hypertension et du rein (SKIPOGH)\footnote{Swiss Kidney Project on Genes in Hypertension}. Pour mener à bien ce projet, l'IUMSP collabore avec l'institut Computational Intelligence for Computational Biology (CI4CB) d'Yverdon-les-Bains au sein duquel a été réalisé ce projet d'approfondissement.

	L'institut CI4CB a reçu la tâche d'analyse d'un dataset aux sources vastes: analyses en laboratoire, questionnaires, consultations, etc. L'analyse de ces données implique les challenges d'une analyse exploratoire, soit la définition des techniques, outils et méthodes les plus adéquats pour pouvoir répondre à la problématique de l'IUMSP: mieux comprendre quels sont les rôles des stéroïdes pour l'HTA. 

	Les 40 dimensions de l'analyse multivariée à effectuer demandent l'utilisation d'outils au paramétrage non trivial pour résoudre la problématique citée ci-dessus.

	La méthode choisie pour cette étude sera divisée en deux parties. Dans un premier temps, nous observerons les corrélations de manière globale grâce à une matrice de nuages de points. Puis, une étude des corrélations partielles sera effectuée en utilisant des cartes auto-organisées (SOM) et leurs composants planaires. Finalement une coloration des SOM sera faite en selon une sélection de variables étrangères: l'âge, le sexe, le BMI, la pression artérielle diastolique et systolique.

	%TODO : Mettre à jour la dernière phrase une fois que le travail aura été fait.